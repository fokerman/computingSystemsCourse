% === T22 - Sistemas de Archivos ===
% David Alejandro Gonzalez Marquez
% fokerman@gmail.com
% https://github.com/fokerman/computingSystemsCourse

\documentclass[aspectratio=169]{beamer}
\usepackage{../packages}

\usepackage{keystroke}
\usepackage{menukeys}

\title{\Huge Sistemas de Archivos}
\author{David Alejandro González Márquez}
\input{../university}
\date{}

\begin{document}

\begin{frame}[plain]
    \titlepage
    \begin{textblock}{100}(30,80)
    \begin{tcolorbox}[size=small,width=\textwidth,colback={gray!30},title={}]
    \begin{center}
     \scriptsize Clase disponible en: \url{https://github.com/fokerman/computingSystemsCourse}
    \end{center}
    \end{tcolorbox}
    \end{textblock}
%     \begin{textblock}{140}(10,70)
%     \textcolor{rojo}{
%     \textbf{Atención}: La clase será grabada por el anfitrión para su posterior y eventual uso académico dentro de nuestra institución. Su participación en la clase implica brindar su consentimiento para participar en la grabación, aunque pueden mantener su video apagado.}
%     \end{textblock}
\end{frame}

\begin{frame}[fragile]{Sistemas de Archivos}
    Los sistemas de archivos son una parte fundamental de un Sistema Operativo.\\
    \bigskip
    Proveen una \textbf{interfaz} lógica al \textbf{almacenamiento secundario}.\\
    \bigskip
    Vamos a distinguir conceptos como\\
    \begin{center}
    \emph{archivo}, \emph{directorio}, \emph{permisos},\\
    \bigskip
    \emph{sistema de archivos}, \emph{organización lógica},\\
    \bigskip
    \emph{organización física} e \emph{implementación}.
    \end{center}
\end{frame}

\begin{frame}[fragile]{Concepto de Archivo}
    Un archivo es una \textbf{unidad lógica} de almacenamiento. Agrupa información bajo un nombre.\\
    Desde el punto de vista del usuario es la \textbf{mínima unidad almacenable} en un sistema de archivos.\\
    \bigskip
    \pause
    Los archivos son una secuencia ordenada de bytes.\\
    Pueden ser interpretados como \textcolor{verdeuca}{texto} o \textcolor{verdeuca}{binarios}.\\
    Su estructura puede ser de caracteres, líneas o registros, ya sea variables o fijos.\\
    \bigskip
    Existen múltiples formatos de archivos, cada con su propio formato interno.
\end{frame}

\begin{frame}[fragile,t]{Atributos y características de los archivos}
    Los archivos tienen asociada información que se almacena como metadata dentro del sistema de archivos.
    { \small
    \begin{itemize}
    \setlength\itemsep{0px}
    \item Nombre: \textcolor{gray}{Nombre simbólico del archivo para lectura humana.}
    \item Identificador: \textcolor{gray}{Valor único de identificación dentro del sistema.}
    \item Tipo: \textcolor{gray}{Información de las características del archivos (ej. es un directorio).}
    \pause
    \item Ubicación: \textcolor{gray}{Puntero al lugar donde se encuentra el archivo.}
    \item Tamaño: \textcolor{gray}{Cantidad de bytes, o bloques del archivo.}
    \item Protección: \textcolor{gray}{Información de control de accesos.}
    \item Fecha/hora: \textcolor{gray}{Información sobre fecha y hora, creación, último acceso y modificación.}
    \end{itemize}
    }
    \pause
    Pueden tener una extensión que permite identificar de que tipo de archivo se trata.\\
    Esta extensión es parte del nombre simbólico del archivo.\\
    \pause
    \bigskip
    \textbf{Número Mágico}: Son secuencias numéricas particulares que se ubican al comienzo del archivo y permiten identificar el tipo de archivo sin usar su extensión {\scriptsize (\url{https://www.filesignatures.net/})}.
\end{frame}

\begin{frame}[fragile,t]{Operaciones sobre un Archivo}
    Los sistemas de archivos cuentan con un conjunto de \textbf{primitivas básicas} para operar con archivos.
    \textcolor{gray}{En todos los casos es necesario llegar al archivo en cuestión para operar con este.}
    \begin{itemize}
    \pause
    \setlength\itemsep{0.3cm}
    \item \textcolor{naranjauca}{Creación de un archivo:}
    Implica encontrar un lugar libre donde ubicar el archivo y completar una entrada en el directorio que permita llegar al mismo.
    \pause
    \item \textcolor{naranjauca}{Escritura de un archivo:}
    Al escribir en un archivo el sistema guarda el puntero al último byte escrito para poder continuar escribiendo desde ahí.
    \pause
    \item \textcolor{naranjauca}{Lectura de un archivo:} Se guarda un puntero al bloque actual que se esta leyendo, este debe ser independiente entre múltiples procesos que accedan al archivo.
    \pause
    \item \textcolor{naranjauca}{Reposicionamiento dentro de un archivo:}
    Operación que permite mover relativa o absolutamente el puntero dentro del archivo.
    \pause
    \item \textcolor{naranjauca}{Borrado de un archivo:} Consiste en encontrar la entrada de archivo y liberar todo su contenido. Borrando también su entrada en el directorio.
    \end{itemize}
    %  \item \textcolor{naranjauca}{Truncado de un archivo:}
    % - Adiccion (append)
    % - Renombrado
\end{frame}

\begin{frame}[fragile,t]{Tabla de archivos abiertos}
    La mayoría de los sistemas requieren para operar con archivos, \textbf{abrirlos} y al terminar \textbf{cerrarlos}.\\
    \begin{itemize}
    \item \texttt{open()}: Abre un archivo para ser utilizando, tanto como lectura o escritura.
    \item \texttt{close()}: Cierra un archivo abierto.
    \end{itemize}
    \pause
    Los archivos que son abiertos por un proceso, son registrados en una \textbf{Tabla de archivos abiertos}.
    La tabla contiene la información necesaria para poder acceder al archivo sin necesidad de buscar nuevamente en el directorio.\\
    \bigskip
    \pause
    Por cada archivo abierto tenemos: { \small
    \begin{itemize}
    \setlength\itemsep{0px}
    \item Puntero al archivo: \textcolor{gray}{Registra la ubicación donde realizar la próxima lectura o escritura.}
    \item Contador de aperturas: \textcolor{gray}{Registra cuantos procesos tiene abierto el archivo.}
    \item Ubicación dentro del disco: \textcolor{gray}{Información para ubicar el archivo en el disco.}
    \item Derechos de acceso: \textcolor{gray}{Modo de apertura del archivo, lectura o escritura.}
    \end{itemize} }
    \small Algunos sistemas proveen mecanismos de \textbf{bloqueo compartido} y \textbf{bloqueo exclusivo} sobre los archivos.
\end{frame}

\begin{frame}[fragile,t]{Métodos de acceso a archivos}
    \begin{itemize}
    \item \textcolor{naranjauca}{Acceso secuencial}\\
    {\small La información \textbf{se procesa en orden}, un registro o dato dentro del archivo después del otro. Este modo es el mas común para la mayoría de los programas. Para implementarlo se requiere guardar cuanto se leyó o escribió del archivo en cada operación y mover el puntero en consecuencia.}\\
    \bigskip
    \begin{center}
    \includegraphics[scale=1.3]{img/acceso_secuencial-layer1.pdf}\hspace{1cm}
    \includegraphics[scale=1.3]{img/acceso_secuencial-layer2.pdf}
    \end{center}
    \bigskip
    \pause
    \item \textcolor{naranjauca}{Acceso directo o relativo}\\
    {\small Este método supone a un archivo como una secuencia de registros lógicos de longitud fija. Utilizando la posibilidad de \textbf{acceder aleatoriamente} a cualquier bloque del disco. Es posible identificar cualquier registro dentro de los bloques del sistema de archivos y acceder a estos sin ninguna restricción en cuanto al orden de lectura o escritura.}
    \end{itemize}
\end{frame}

\begin{frame}[fragile,t]{Estructura de almacenamiento}
    Un disco o cualquier dispositivos de almacenamiento secundario, puede ser utilizado para albergar un único sistema de archivos, o múltiples sistemas de archivos de distintos tipos.\\
    \bigskip
    Las partes en las que se divide el disco se conocen como \textbf{particiones}.\\
    Estas pueden combinarse para generar espacios de mayor tamaño denominados \textbf{volumenes}.\\
    \bigskip
    \pause
    Cada volumen que contenga un sistema de archivos cuenta con un \textbf{directorio de contenidos} o simplemente directorio,
    que contiene la información sobre los archivos almacenados.\\
    \bigskip
    La estructura lógica de un directorio nos debe permitir:
    \begin{itemize}
    \setlength\itemsep{0px}
    \item \textcolor{verdeuca}{Listar un directorio y buscar un archivo}
    \item \textcolor{verdeuca}{Agregar, eliminar y renombrar un archivo}
    \item \textcolor{verdeuca}{Recorrer el sistema de archivos}
    \end{itemize}
%         \begin{textblock}{76}(10,15) 
%          \includegraphics[scale=0.2]{img/partition.png}
%         \end{textblock}
\end{frame}

\begin{frame}[fragile,t]{Estructura de directorios}
    Directorio de un nivel único\\
    \begin{textblock}{140}(10,14)
    \only<1->{
    \begin{center}
    \includegraphics[scale=1.8]{img/directorios-layer1.pdf}
    \end{center}
    }
    \end{textblock}
    \begin{textblock}{140}(10,45)
    \only<2->{
    Directorio de dos niveles\\
    \begin{center}
    \includegraphics[scale=1.8]{img/directorios-layer2.pdf}
    \end{center}
    }
    \end{textblock}
\end{frame}

\begin{frame}[fragile,t]{Estructura de directorios}
    Directorio con estructura de árbol\\
    \begin{textblock}{140}(10,14)
    \begin{center}
    \includegraphics[scale=1.8]{img/directorios-layer3.pdf}
    \end{center}
    \end{textblock}
\end{frame}

\begin{frame}[fragile,t]{Estructura de directorios}
    Directorios en un grafo acíclico\\
    \begin{textblock}{140}(10,14)
    \begin{center}
    \includegraphics[scale=1.8]{img/directorios-layer4.pdf}
    \end{center}
    \end{textblock}
\end{frame}

\begin{frame}[fragile,t]{Montaje de sistemas de archivos}
    Montar un sistema de archivos dentro de otro consiste en colocar en algún punto del árbol de directorios, todo el contenido lógico de otro sistema de archivos.
    \begin{textblock}{140}(40,25) \only<1->{\includegraphics[scale=2]{img/montado-layer1.pdf}} \end{textblock}
    \begin{textblock}{140}(40,25) \only<2->{\includegraphics[scale=2]{img/montado-layer2.pdf}} \end{textblock}
    \begin{textblock}{140}(40,25) \only<3->{\includegraphics[scale=2]{img/montado-layer3.pdf}} \end{textblock}
    \begin{textblock}{140}(40,25) \only<4->{\includegraphics[scale=2]{img/montado-layer4.pdf}} \end{textblock}
    \begin{textblock}{140}(10,65)
    Ejemplo:\\ \small
    \textcolor{gray}{En el sistema de arhivos presentado se tiene montado en \texttt{/users} un sistema de archivos, 
    mientras que en \texttt{/data/pictures} y \texttt{/data/today} se tiene otros dos sistemas de archivos diferentes montados.\\
    Considerar que no necesariamente todos tiene el mismo tipo de sistema de archivos.}\\
    \end{textblock}
\end{frame}

% \begin{frame}[fragile,t]{Compartición de archivos}
% \end{frame}

% \begin{frame}[fragile,t]{Semantica de coherencia}
% - Semantica UNIX
% - Semantica de sesion
% - Semantica de archivos compartidos inmutables
% \end{frame}

\begin{frame}[fragile,t]{Protección}
    Vamos a hablar de protección desde el punto de vista de la seguridad de la información.\\
    \bigskip
    \pause
    Se plantean dos mecanismos:\\
    \begin{itemize}
    \item \textcolor{naranjauca}{Tipos de acceso}\\
    Proporciona un acceso controlado dependiendo del tipo de acción a realizar. Limitando lecturas, escrituras, ejecución, borrado o listado.
    \pause
    \item \textcolor{naranjauca}{Control de acceso}\\
    Consiste en hacer que el acceso dependa de la identidad del usuario. Se asocia a cada archivo y directorio una lista de control de acceso que especifica que usuarios y con que permisos pueden acceder.
    \end{itemize}
    \bigskip
    \textcolor{verdeuca}{Esta última técnica tiene el problema que la lista de control de accesos puede ser muy grande.}\\
    Para reducir el tamaño muchos sistemas clasifican a los usuarios en grupos de usuarios.
\end{frame}

\begin{frame}[fragile,t]{Protección}
    En los sistemas UNIX se utilizan 3 bits para identificar que permisos se tiene sobre el archivo.\\
    \begin{itemize}
    \item \textcolor{verdeuca}{\texttt{r}}: Permite lectura
    \item \textcolor{verdeuca}{\texttt{w}}: Permite escritura
    \item \textcolor{verdeuca}{\texttt{x}}: Permite ejecución
    \end{itemize}
    \pause
    Estos permisos dependen del usuario:
    \begin{itemize}
    \item \textcolor{verdeuca}{Propietario} (dueño o creador del archivo)
    \item \textcolor{verdeuca}{Grupo} (conjunto de usuarios que comparten el archivo)
    \item \textcolor{verdeuca}{Universo} (todos)
    \end{itemize}
    \pause
    Luego un archivo tiene la siguiente secuencia de permisos: \textcolor{verdeuca}{\texttt{rwx}\texttt{rwx}\texttt{rwx}}\\
    \bigskip
    Donde las primeras tres letras son del Propietario, las siguientes del Grupo y las últimas para el resto de los usuarios.
\end{frame}

\begin{frame}[fragile,t]{Implementación de sistemas de archivos}
    Implementar un sistema de archivos propone dos problemas:
    \bigskip
    \pause
    \begin{itemize}
    \setlength\itemsep{0.4cm}
    \item Definir que aspecto debe tener para el usuario, su \textbf{interfaz}. Definir un archivo y sus atributos, las operaciones permitidas y la estructura de directorios utilizada para organizarlos.
    \item Crear \textbf{algoritmos y estructuras de datos} que permitan mapear el sistema lógico de archivos sobre los dispositivos físicos de almacenamiento secundario.
    \end{itemize}
    \bigskip
    \pause
    Los sistemas de almacenamiento secundario como discos, almacenan la información en \textbf{sectores} que agrupan un conjunto de bytes (usualmente 512 bytes).\\
    \bigskip
    Las transferencias entre memoria y disco se realizan de a \textbf{bloques} que es la mínima unidad de lectura/escritura para un sistema de archivos.
\end{frame}

\begin{frame}[fragile,t]{Estructura de un sistema de archivos}
    Los sistemas de archivos están compuestos por varios niveles de abstracción:
    \bigskip
    \pause
    \begin{itemize}
    \setlength\itemsep{0.2cm}
    \item \textbf{Control de E/S}:
    Controladores de dispositivos y rutinas de atención de interrupciones, encargadas de transferir bloques entre la memoria y el almacenamiento secundario.
    \pause
    \item \textbf{Sistema básico de archivos}:
    Se ocupa de enviar comandos genéricos al controlador apropiado a fin de leer o escribir bloques físicos dentro del disco.
    \pause
    \item \textbf{Módulo de organización de archivos}:
    Tiene información sobre los bloques lógicos y físicos del archivo. Los bloques lógicos se numeran de 0 a N, mientras que los bloques físicos corresponden a la ubicación en disco. Se ocupa de realizar esta traducción y además de gestionar los bloques libres y ocupados.
    \pause
    \item \textbf{Sistema lógico de archivos}:
    Gestiona la información sobre los archivos (metadata). Cada archivo tiene asociado un FCB (file control block) que contiene la ubicación de los datos, permisos y propietario.
    \end{itemize}
\end{frame}

\begin{frame}[fragile,t]{Implementación de sistemas de archivos}
    Un sistema de archivos en un disco físico puede contener información particular en bloques distinguidos.\\
    \bigskip
    \begin{itemize}
    \setlength\itemsep{0.4cm}
        \item \textbf{Bloque de control de arranque} (por cada volumen): Usualmente es el primer bloque del sistema, puede contener información que permita inicializar el sistema operativo. Se lo conoce como sector de arranque.
    \item \textbf{Bloque de control de volumen} (por cada volumen): Contiene detalles sobre el volumen o partición, como el número total de bloques, donde comienza y donde termina. Cantidad de bloques libres y ocupados. Se lo puede encontrar como superblock en sistemas Unix.
    \item \textbf{Estructura de directorios}: Para cada sistema de archivos tiene su propia estructura de directorios.
    \end{itemize}
\end{frame}

\begin{frame}[fragile]{Implementación de directorios}
    \begin{itemize}
    \setlength\itemsep{0.5cm}
    \item \textcolor{naranjauca}{Lista lineal}\\
    Es el método más simple. Consiste en una lista de nombres de archivos con punteros a los bloques de datos.
    Su desventaja es que requiere mucho tiempo de ejecución para todas las operaciones.
    Es posible mejorar el rendimiento de esta solución utilizando un árbol ordenado a diferencia de una lista.
    \pause
    \item \textcolor{naranjauca}{Tabla de hash}\\
    Con este método se utiliza una lista lineal, pero adicionalmente se tiene una tabla de hash.
    Esta tabla permite a partir de un valor calculado sobre el nombre del archivo, encontrar el puntero al bloque de datos del mismo.
    \end{itemize}
\end{frame}

\begin{frame}[fragile,t]{Métodos de asignación}
    \begin{textblock}{76}(10,15) 
    \textbf{Asignación contigua}\\
    Requiere que cada archivo ocupe un conjunto contiguo de bloques en disco.\\
    Proporciona un acceso muy eficiente, ya que solo se requiere conocer el bloque actual para identificar al siguiente.\\
    El problema de este método es la asignación de un nuevo espacio, ya que se puede sufrir de \emph{fragmentación externa}.
    \end{textblock}
    \begin{textblock}{100}(90,15) \includegraphics[scale=0.5]{img/asignacion_contigua_enlazada_indexada-layer1.pdf} \end{textblock}
\end{frame}

\begin{frame}[fragile,t]{Métodos de asignación}
    \begin{textblock}{76}(10,15) 
    \textbf{Asignación entrelazada}\\
    El directorio contiene un puntero al primer bloque, y luego cada bloque contiene punteros al siguiente bloque del archivo.\\
    Este método es eficiente para acceso secuencia, pero resulta lento para el acceso directo.\\
    Los punteros ocupan espacio, que puede llegar a no ser despreciable.\\
    Además, es poco resistente a errores, ya que perder un puntero implica perder la continuación del archivo.
    \end{textblock}
    \begin{textblock}{100}(90,15) \includegraphics[scale=0.5]{img/asignacion_contigua_enlazada_indexada-layer2.pdf} \end{textblock}
\end{frame}

\begin{frame}[fragile,t]{Métodos de asignación}
    \begin{textblock}{76}(10,15) 
    \textbf{Asignación indexada}\\
    Resuelve los problema de la asignación entrelazada, agrupando todos los punteros en un solo bloque, denominado bloque de índice.\\
    Los sistemas de archivos utilizan este método, agregando esquemas entrelazados e indices multinivel.
    \end{textblock}
    \begin{textblock}{100}(90,15) \includegraphics[scale=0.5]{img/asignacion_contigua_enlazada_indexada-layer3.pdf} \end{textblock}
\end{frame}

\begin{frame}[fragile,t]{Recuperación}
    Los sistemas de archivos mantienen información tanto en almacenamiento secundario como en memoria.\\
    Ante un evento de fallo del sistema, se pueden producir perdidas de datos o incluso dejar datos incoherentes.\\
    \bigskip
    \pause
    \textbf{Comprobación de coherencia}\\
    \bigskip
    Para detectar y corregir estos problemas, existen programas que realizan un chequeo de integridad sobre los datos en el disco.
    Este busca incoherencias entre los metadatos y las trata de reparar. 
\end{frame}


\begin{frame}[fragile,t]{Sistemas de archivos con estructura de registro (\emph{journaling})}
    Este tipo de sistemas utilizan algoritmos de recuperación basados en \textbf{registros de transacciones}.
    Ante eventos de fallos donde se pierde información puede no ser posible recuperar datos y los chequeos de coherencia ser insuficientes.\\
    \bigskip
    \pause
    Su funcionamiento se basa en \textcolor{naranjauca}{\textbf{escribir secuencialmente en un registro}}
    todos los cambios sobre la metadata de los archivos.
    \textcolor{verdeuca}{Se denomina transacción a un conjunto cambios para realizar una tarea específica.}\\
    \bigskip
    \pause
    Una vez escritos los cambios en el registro, la operación se puede dar por terminada y confirmada.
    \textcolor{verdeuca}{Mientras tanto, las operaciones del registro son leidas y \textbf{aplicadas sobre las estructuras reales del sistema}.}
    Se lleva seguimiento de cual operación fue realizada y cual no.\\
    \bigskip
    \pause
    Una vez que se terminan de realizar todas las operaciones de una transacción, la entrada en el registro es borrada, dejando lugar en el buffer para nuevos cambios.
\end{frame}

\begin{frame}[fragile]
    \frametitle{Bibliografía}
    \begin{itemize}
        \setlength\itemsep{0.5cm}
        \item[-] \small Silberschatz, ``Fundamentos de Sistemas Operativos'', 7ma Edición, 2006.\\
        \begin{itemize}
            \item \textbf{Capítulo 10 - Interfaz del sistema de archivos}, páginas 333-361
            \item \textbf{Capítulo 11 - Implementación de sistemas de archivos}, páginas 369-386 y 392-393
        \end{itemize}
        \item[-] \small Silberschatz, ``Operating System Concepts'', 10th Edition, 2018.\\
        \begin{itemize}
            \item \textbf{Chapter 13 - File-System Interface}, páginas 529-555
            \item \textbf{Chapter 14 - File-System Implementation}, páginas 564-578 y 586-589
            \item \textbf{Chapter 15 - File-System Internals}, páginas 598-601
        \end{itemize}
    \end{itemize}
\end{frame}

\begin{frame}[plain]
    \begin{center}
    \vspace{2cm}
    \huge ¡Gracias!\\
    \vspace{2cm}
    \normalsize Recuerden leer los comentarios adjuntos\\ en cada clase por aclaraciones.
    \end{center}
\end{frame}

\end{document}
