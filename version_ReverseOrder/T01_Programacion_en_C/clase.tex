% === T01 - Programación en C ===
% David Alejandro Gonzalez Marquez
% fokerman@gmail.com
% https://github.com/fokerman/computingSystemsCourse

\documentclass[aspectratio=169]{beamer}
\usepackage{../packages}

\title{\Huge Programación en C}
\author{David Alejandro González Márquez}
\institute{}

\date{}

\begin{document}

\begin{frame}[plain]
    \titlepage
    \begin{textblock}{100}(30,80)
    \begin{tcolorbox}[size=small,width=\textwidth,colback={gray!30},title={}]
    \begin{center}
     \scriptsize Clase disponible en: \url{https://github.com/fokerman/computingSystemsCourse}
    \end{center}
    \end{tcolorbox}
    \end{textblock}
%     \begin{textblock}{140}(10,70)
%     \textcolor{rojo}{
%     \textbf{Atención}: La clase será grabada por el anfitrión para su posterior y eventual uso académico dentro de nuestra institución. Su participación en la clase implica brindar su consentimiento para participar en la grabación, aunque pueden mantener su video apagado.}
%     \end{textblock}
\end{frame}

% \begin{frame}[fragile,t]{}
% - Diferencias entre C y Pythonn
% \end{frame}

\begin{frame}[fragile]{El lenguaje C}
    Desarrollado por \textbf{Dennis Ritchie} en los Laboratorios Bell (1973).\\
    \bigskip
    Tiene tipos de datos \textbf{estáticos}, débilmente tipado.\\
    \bigskip
    Diseñado para la programación estructurada de \textbf{alto nivel}\\
    pero con soporte para operar a \textbf{bajo nivel}.\\
    \pause
    \bigskip
    Estandarización del lenguaje \texttt{ANSI} \texttt{C}, posteriormente\\
    adoptado por \texttt{ISO} (\texttt{C89}, \texttt{C90}, \texttt{C99}, \texttt{C11}).\\
    \bigskip
    La adopción del estándar lo hace \textbf{portable} entre plataformas y arquitecturas.\\
\end{frame}

\begin{frame}[fragile,t]{Hola Mundo}

    \begin{verbatim}
    #include <stdio.h>

    int main() {
        printf("Hola mundo\n");
        return 0;
    }
    \end{verbatim}
    \small
    \pause
    \textcolor{naranjauca}{\texttt{\#include <stdio.h>}}: Incluye una biblioteca de funciones para escribir en entrada/salida.\\
    \bigskip
    \pause
    \textcolor{naranjauca}{\texttt{int main()}}: Declara una función principal que va a ser nuestro programa.\\
    \bigskip
    \pause
    \textcolor{naranjauca}{\texttt{printf}}: Función para imprimir en pantalla con un formato específico.\\
    \bigskip
    \pause
    \textcolor{naranjauca}{\texttt{return 0;}}: Termina la función principal y sale del programa con un cero.\\
\end{frame}

\begin{frame}[fragile,t]{Declaración de Variables}
    \begin{textblock}{73}(80,4)
    \begin{block}{\small Sintaxis}
    \small
    \fbox{\vphantom{fg} \texttt{tipo-de-variable}} \hspace{0.2cm} \fbox{\vphantom{fg} \texttt{nombre-de-variable}} {\large \texttt{;}}
    \end{block}
    \end{textblock}
    \begin{textblock}{80}(90,20)
    \begin{tabular}{l|c}
    \small
    \textcolor{naranjauca}{Tipo} & \textcolor{naranjauca}{Descripción}\\
    \hline
    \texttt{int}    & \textcolor{verdeuca}{entero de 4 bytes}      \\
    \texttt{char}   & \textcolor{verdeuca}{entero de 1 byte}       \\ 
    \texttt{float}  & \textcolor{verdeuca}{punto flotante 32 bits} \\
    \texttt{double} & \textcolor{verdeuca}{punto flotante 64 bits} \\
    \end{tabular}
    \end{textblock}
    \begin{textblock}{60}(8,18)
    \small
    Las \textbf{variables} se declaran y se asignan mediante el operador igual (\texttt{=}).\\
    \bigskip
    Toda variable ocupa una determinada cantidad de bytes dada por su tipo.\\
    \end{textblock}
    \begin{textblock}{100}(5,50)
    \textcolor{gray}{Ejemplos:}\\
    \bigskip
    \begin{tabular}{ll}
    \normalsize \verb|int i;|               & \uncover<2->{\textcolor{verdeuca}{$\leftarrow$ \small Variable entera de 4 bytes de nombre \texttt{i}}\\}
    \normalsize \verb|int pepe = 35;|       & \uncover<3->{\textcolor{verdeuca}{$\leftarrow$ \small Variable y asignación de un valor}\\}
    \normalsize \verb|char unaLetra = 'a';| & \uncover<4->{\textcolor{verdeuca}{$\leftarrow$ \small Variable de un byte y asignación de un carácter}\\}
    \normalsize \verb|char contador = 9;|   & \uncover<5->{\textcolor{verdeuca}{$\leftarrow$ \small Variable de un byte y asignación de un número}\\}
    \normalsize \verb|float numero = 5.3;|  & \uncover<6->{\textcolor{verdeuca}{$\leftarrow$ \small Variable en punto flotante de 32 bits}\\}
    \end{tabular}
    \end{textblock}
\end{frame}

\begin{frame}[fragile,t]{Operadores}
    \begin{textblock}{50}(10,15) \small
     Las \textbf{expresiones} del lenguaje se construyen mediante \textbf{operadores} matemáticos y lógicos.\\ \bigskip
     En C las expresiones resultan en valores numéricos.\\ \bigskip
     El valor cero es interpretado como el valor lógico \textbf{\texttt{falso}}, mientras que un valor distinto de cero se interpreta como \textbf{\texttt{verdadero}}.
    \end{textblock}
    \begin{textblock}{80}(61,4)
    \begin{tabular}{c|p{2.3cm}}
    \small \textcolor{naranjauca}{Operador} & \small \textcolor{naranjauca}{Descripción}\\
    \hline
    \verb|+|  & \small \textcolor{verdeuca}{suma}           \\
    \verb|-|  & \small \textcolor{verdeuca}{resta}          \\
    \verb|*|  & \small \textcolor{verdeuca}{multiplicación} \\
    \verb|/|  & \small \textcolor{verdeuca}{división}       \\
    \verb|%|  & \small \textcolor{verdeuca}{resto de división}\\ 
    \hline
    \verb|==| & \small \textcolor{verdeuca}{igual}          \\
    \verb|!=| & \small \textcolor{verdeuca}{distinto}       \\
    \verb|>|  & \small \textcolor{verdeuca}{mayor}          \\
    \verb|>=| & \small \textcolor{verdeuca}{mayor o igual}    \\
    \verb|<|  & \small \textcolor{verdeuca}{menor}          \\
    \verb|<=| & \small \textcolor{verdeuca}{menor o igual}    \\
    \end{tabular}
    \end{textblock}
    \begin{textblock}{80}(109,4)
    \begin{tabular}{c|p{2.15cm}}
    \small \textcolor{naranjauca}{Operador} & \small \textcolor{naranjauca}{Descripción}\\
    \hline
    \verb|<<| & \small \textcolor{verdeuca}{shift a izquierda} \\
    \verb|>>| & \small \textcolor{verdeuca}{shift a derecha}   \\
    \hline
    \verb|!|  & \small \textcolor{verdeuca}{not lógico}     \\
    \verb|&&| & \small \textcolor{verdeuca}{and lógico}     \\
    \verb.||. & \small \textcolor{verdeuca}{or lógico}      \\
    \hline
    \textasciitilde  & \small \textcolor{verdeuca}{not bit a bit}  \\
    \verb|&|  & \small \textcolor{verdeuca}{and bit a bit}  \\
    \verb.|.  & \small \textcolor{verdeuca}{or bit a bit}   \\
    \verb|^|  & \small \textcolor{verdeuca}{xor bit a bit}  \\
    \end{tabular}
    \end{textblock}
    \begin{textblock}{100}(10,68)
    \only<2->{
    \small
    \textcolor{gray}{Ejemplos:}\\
    \hspace{1cm} \texttt{int valor = (2 << 10) - (n + m) / 5;}\\
    \hspace{1cm} \texttt{int cond  = (b <= 6 || c > 3) \&\& !(a == 5);}\\
    \hspace{1cm} \texttt{int stat  = (} \textasciitilde \texttt{0x111 | v) \& 0x5555;}\\ %$~$
    }
    \end{textblock}
\end{frame}

\begin{frame}[fragile,t]{Declaración de Funciones}
    \begin{textblock}{100}(57,4)
    \begin{block}{\small Sintaxis}
    \vspace{0.2cm}
    \scriptsize
    \fbox{\vphantom{fg} \texttt{tipo-de-retorno}} \hspace{0.2cm} \fbox{\vphantom{fg} \texttt{nombre-de-función}} {\large \texttt{(}} \fbox{\vphantom{fg} \texttt{parametros-de-la-función} } {\large \texttt{)}}
    {\large \texttt{\{}}\\ \vspace{0.2cm}
    \hspace{3.19cm} \fbox{$\cdots$ \texttt{cuerpo-de-la-función} $\cdots$ } {\large \texttt{;}}\\ \vspace{0.2cm}
    \hspace{2cm} {\large \texttt{\}}}
    \end{block}
    \end{textblock}
    \begin{textblock}{45}(5,15)
    Las funciones pueden tener cualquier nombre, toman una lista de \textbf{parámetros} y solo pueden \textbf{retornar} un valor resultado.\\
    \bigskip
    Existe una función distingida de nombre \textbf{\texttt{main}} que es la principal en un programa C.\\
    \bigskip
    Para llamar a una función se usa su nombre y se pasan los parámetros entre paréntesis.\\
    El resultado se puede asignar o no a una variable.\\
    \end{textblock}
    \begin{textblock}{100}(57,40)
    \textcolor{gray}{Ejemplos:}\\
    \bigskip
    \normalsize \verb|int suma(int a, int b) {|\\
    \normalsize \verb|    int s = a + b;|\\
    \normalsize \verb|    return s;|\\
    \normalsize \verb|}|\\
    \bigskip
    \normalsize \verb|int esPar(int numero) {|\\
    \normalsize \verb|    return !(numero & 1);|\\
    \normalsize \verb|}|\\
    \end{textblock}
    \begin{textblock}{100}(116,50)
    \normalsize \verb|int main() {|\\
    \normalsize \verb|    int t;|\\
    \normalsize \verb|    t = suma(2,4);|\\
    \normalsize \verb|    return esPar(t);|\\
    \normalsize \verb|}|\\
    \end{textblock}
\end{frame}

\begin{frame}[fragile,t]{Estructuras de control\\ (\texttt{for})}
    \begin{textblock}{90}(64,4)
    \begin{block}{\small Sintaxis}
    \vspace{0.2cm}
    \scriptsize
    {\large \texttt{for}} {\large \texttt{(}}
    \fbox{\vphantom{fg} \texttt{condición-comienzo}} {\large \texttt{;}}
    \fbox{\vphantom{fg} \texttt{condición-final}} {\large \texttt{;}}
    \fbox{\vphantom{fg} \texttt{incremento}} {\large \texttt{)}} {\large \texttt{\{}}\\ \vspace{0.2cm}
    \hspace{2cm} \fbox{$\cdots$ \texttt{cuerpo-del-ciclo} $\cdots$ } {\large \texttt{;}}\\ \vspace{0.2cm}
    \hspace{1cm} {\large \texttt{\}}}
    \end{block}
    \end{textblock}
    \begin{textblock}{50}(5,20)
    \small
    La sentencia \texttt{condición-comienzo} se ejecuta una vez al comienzo del ciclo,\\
    \bigskip
    la \texttt{condición-final} se ejecuta en cada iteración hasta hacerse falsa,\\
    \bigskip
    y la sentencia de \texttt{incremento} se ejecuta al final de cada iteración.
    \end{textblock}
    \begin{textblock}{100}(64,40)
    \textcolor{gray}{Ejemplo:}\\
    \bigskip
    \normalsize \verb|int potenciaDe2(int n) {|\\
    \normalsize \verb|    int t = 1;|\\
    \normalsize \verb|    for(int i=0; i<n; i++) {|\\
    \normalsize \verb|          t = t * 2;|\\
    \normalsize \verb|    }|\\
    \normalsize \verb|    return t;|\\
    \normalsize \verb|}|\\
    \end{textblock}
\end{frame}

\begin{frame}[fragile,t]{Estructuras de control\\ (\texttt{while})}
    \begin{textblock}{45}(60,4)
    \begin{block}{\small Sintaxis}
    \vspace{0.2cm}
    \scriptsize
    {\large \texttt{while}} {\large \texttt{(}}
    \fbox{\vphantom{fg} \texttt{condición}} {\large \texttt{)}} {\large \texttt{\{}}\\ \vspace{0.2cm}
    \hspace{0.7cm} \fbox{$\cdots$ \texttt{cuerpo-del-while} $\cdots$ } {\large \texttt{;}}\\ \vspace{0.2cm}
    \hspace{0.3cm} {\large \texttt{\}}}
    \end{block}
    \end{textblock}
    \begin{textblock}{45}(110,4)
    \uncover<2>{
    \begin{block}{\small Sintaxis do-while}
    \vspace{0.2cm}
    \scriptsize
    {\large \texttt{do}} {\large \texttt{\{}}\\ \vspace{0.2cm}
    \hspace{0.7cm} \fbox{$\cdots$ \texttt{cuerpo-del-while} $\cdots$ } {\large \texttt{;}}\\ \vspace{0.2cm}
    {\large \texttt{\}}}
    {\large \texttt{while}} {\large \texttt{(}} \fbox{\vphantom{fg} \texttt{condición}} {\large \texttt{);}}
    \end{block}
    }
    \end{textblock}
    \begin{textblock}{50}(5,20)
    \small
    Mientras la sentencia \texttt{condición}\\ se cumpla, es decir, sea vedadera,\\
    se ejecutará el \texttt{cuerpo-del-while}.\\
    \bigskip
    Dentro de \texttt{cuerpo-del-while} se\\ deben alterar el contexto para que\\ la \texttt{condición} sea falsa y se pueda terminar el ciclo.
%     \bigskip
%     y la sentencia de \texttt{incremento} se ejecuta al final de cada iteración.
    \end{textblock}
    \begin{textblock}{100}(64,40)
    \textcolor{gray}{Ejemplo:}\\
%     \bigskip
    \normalsize \verb|int potenciaDe2(int n) {|\\
    \normalsize \verb|    int t = 1;|\\
    \normalsize \verb|    int i = 0;|\\
    \normalsize \verb|    while( i != n ) {|\\
    \normalsize \verb|          t = t * 2;|\\
    \normalsize \verb|          i++;|\\
    \normalsize \verb|    }|\\
    \normalsize \verb|    return t;|\\
    \normalsize \verb|}|\\
    \end{textblock}
\end{frame}

\begin{frame}[fragile,t]{Estructuras de control\\ (\texttt{if})}
    \begin{textblock}{60}(80,4)
    \begin{block}{\small Sintaxis}
    \vspace{0.2cm}
    \scriptsize
    {\large \texttt{if}} {\large \texttt{(}} \fbox{\vphantom{fg} \texttt{condición}} {\large \texttt{)}} {\large \texttt{\{}}\\ \vspace{0.2cm}
    \hspace{2cm} \fbox{$\cdots$ \texttt{bloque-verdadero} $\cdots$ } {\large \texttt{;}}\\ \vspace{0.2cm}
    \hspace{1cm} {\large \texttt{\} else \{}}\\ \vspace{0.2cm}
    \hspace{2cm} \fbox{$\cdots$ \texttt{bloque-falso} $\cdots$ } {\large \texttt{;}}\\ \vspace{0.2cm}
    \hspace{1cm} {\large \texttt{\}}}
    \end{block}
    \end{textblock}
    \begin{textblock}{50}(5,20)
    \small
    La sentencia \texttt{condición} determina qué bloque de código debe ser ejecutado.\\
    \bigskip
    \texttt{bloque-verdadero} se ejecuta si la condición es verdadera,\\
    \bigskip
    y \texttt{bloque-falso} se ejecuta si la condición es falsa.
    \end{textblock}
    \begin{textblock}{100}(80,48)
    \textcolor{gray}{Ejemplo:}
    \end{textblock}
    \begin{textblock}{100}(98,50)
    \small
    \verb|int collatz(int x) {|\\
    \verb|    int res; |\\
    \verb|    if( esPar(x) ) {|\\
    \verb|      res = x / 2;|\\
    \verb|    } else {|\\
    \verb|      res = 3 * x + 1;|\\
    \verb|    }|\\
    \verb|    return res;|\\
    \verb|}|\\
    \end{textblock}
\end{frame}

\begin{frame}[fragile,t]{Ejemplo}
\begin{textblock}{60}(10,14)
\small
\textcolor{verdeuca}{De los primeros \texttt{n} números enteros,\\ sumar los que no sean múltiplo de 5:}
\vspace{-0.1cm}
\begin{verbatim}
int sumNoMul5(int n) {
    int acum = 0;
    for(int i = 1; i <= n; i++) {
        if(i % 5) {
            acum = acum + i;
        }
    }
    return acum;
}
\end{verbatim}
\end{textblock}
\begin{textblock}{60}(80,14)
\textcolor{verdeuca}{Sumar tres números de 4 bytes y saturar el resultado a 1 byte con signo:}
\vspace{-0.1cm}
\begin{verbatim}
char sumaSaturada(int a, int b, int c) {
    int suma =  a + b + c;
    if(suma > 127) { // 127d = 0x7F
        suma = 127;
    }
    if(suma < -128) { // -128d = 0x80
        suma = -128;
    }
    return suma;
}
\end{verbatim}
\end{textblock}
\end{frame}

% \begin{frame}[fragile,t]{Ejemplo}
% \small
% \textcolor{verdeuca}{Calular el valor de un polinomio de gardo 2:}
% \begin{verbatim}
% float poli2(float x, float a, float b, float c) {
%     return a * x * x + b * x + c;
% }
% \end{verbatim}
% \textcolor{verdeuca}{Aproximar el área de bajo de la curva para el valor absoluto de un polinomio de grado dos:}
% \begin{verbatim}
% float area(float init, float end, float a, float b, float c) {
%     float acum = 0;
%     float interval = 0.1;
%     for(float x = init; x <= end; x = x + interval) {
%         float valor = poli2(x,a,b,c) * interval;
%         if(valor >= 0) {
%             acum = acum + valor;
%         } else {
%             acum = acum - valor;
%         }
%     }
% }
% \end{verbatim}
% \end{frame}

% \begin{frame}[fragile,t]{Ejemplo}
% Calular un polinomio de gardo 2.
% \begin{verbatim}
% float poli2(float x, float a, float b, float c) {
%     float term0 = c;
%     float term1 = b * x;
%     float term2 = a * x * x;
%     float suma = a + b + c;
%     return suma;
% \end{verbatim}
% \end{frame}

\begin{frame}[fragile,t]{Declaración de estructuras de datos}
    \begin{textblock}{87}(68,8)
    \begin{block}{\small Sintaxis}
    \vspace{0.2cm}
    \scriptsize
    {\large \texttt{struct}} \fbox{\vphantom{fg} \texttt{nombre-de-la-estructura}} {\large \texttt{\{}}\\ \vspace{0.2cm}
    \hspace{2cm} \fbox{\vphantom{fg} \texttt{tipo-de-variable-0}} \hspace{0.2cm} \fbox{\vphantom{fg} \texttt{nombre-de-variable-0}} {\large \texttt{;}}\\ \vspace{0.2cm}
    \hspace{2cm} $\cdots$\\ \vspace{0.2cm}
    \hspace{2cm} \fbox{\vphantom{fg} \texttt{tipo-de-variable-n}} \hspace{0.2cm} \fbox{\vphantom{fg} \texttt{nombre-de-variable-n}} {\large \texttt{;}}\\ \vspace{0.2cm}
    \hspace{1cm} {\large \texttt{\};}}
    \end{block}
    \end{textblock}
    \begin{textblock}{55}(5,15)
    \small
    Un \texttt{struct} funciona como un grupo de \textbf{datos ordenados}, o como un tipo de datos compuesto.\\
    \bigskip
    Cada dato dentro del \texttt{struct} tiene un \textbf{tipo} y un \textbf{orden}.\\
    \bigskip
    Los podemos declarar y usar como parámetros de funciones o como variables dentro de una función.\\
    \end{textblock}
    \begin{textblock}{87}(68,50)
    \textcolor{gray}{Ejemplo:}
    \end{textblock}    
    \begin{textblock}{87}(68,52)
    \scriptsize
    \begin{verbatim}
    struct persona {
        char*  nombre;
        int    dni;
        float  altura;
    };
    
    struct persona p;
    p.nombre = "Pepe Lujan";
    p.dni = 32038084;
    p.altura = 1.68;
    \end{verbatim}
    \end{textblock}
    \begin{textblock}{87}(110,52)
    \scriptsize
    \begin{verbatim}
    struct vector {
        double x;
        double y;
        double z;
    };
    
    struct vector v;
    v.x = 8.99;
    v.y = 2.5;
    v.z = 12.31;
    \end{verbatim}
    \end{textblock}
\end{frame}

\begin{frame}[fragile,t]{Arreglos}
    \begin{textblock}{95}(60,4)
    \begin{block}{\small Sintaxis}
    \vspace{0.2cm}
    \scriptsize
    \fbox{\vphantom{fg} \texttt{tipo-del-arreglo}} \hspace{0.2cm} \fbox{\vphantom{fg} \texttt{nombre-del-arreglo}}
    {\large \texttt{[}} \fbox{\vphantom{fg} \texttt{tamaño-del-arreglo}} {\large \texttt{]}} {\large \texttt{;}}\\
    \end{block}
    \end{textblock}
    \begin{textblock}{87}(55,25)
    \textcolor{gray}{Ejemplo:}
    \end{textblock}
    \begin{textblock}{87}(72,30)
    \uncover<2->{ \texttt{int a[5];} }\\
    \uncover<4->{ \texttt{a[0] = 42;} }\\
    \uncover<5->{ \texttt{a[1] = a[0] + 3;} }\\
    \uncover<6->{ \texttt{a[3] = a[0] - a[1];} }\\
    \end{textblock}
    \begin{textblock}{100}(115,24) \only<2->{\includegraphics[scale=0.9]{img/array-layer1.pdf}} \end{textblock}
    \begin{textblock}{100}(115,24) \only<3-3>{\includegraphics[scale=0.9]{img/array-layer2.pdf}} \end{textblock}
    \begin{textblock}{100}(115,24) \only<3-4>{\includegraphics[scale=0.9]{img/array-layer3.pdf}} \end{textblock}
    \begin{textblock}{100}(115,24) \only<3->{\includegraphics[scale=0.9]{img/array-layer4.pdf}} \end{textblock}
    \begin{textblock}{100}(115,24) \only<3-5>{\includegraphics[scale=0.9]{img/array-layer5.pdf}} \end{textblock}
    \begin{textblock}{100}(115,24) \only<3->{\includegraphics[scale=0.9]{img/array-layer6.pdf}} \end{textblock}
    \begin{textblock}{100}(115,24) \only<4->{\includegraphics[scale=0.9]{img/array-layer7.pdf}} \end{textblock}
    \begin{textblock}{100}(115,24) \only<5->{\includegraphics[scale=0.9]{img/array-layer8.pdf}} \end{textblock}
    \begin{textblock}{100}(115,24) \only<6->{\includegraphics[scale=0.9]{img/array-layer9.pdf}} \end{textblock}
    \begin{textblock}{45}(5,15)
    \small
    Los arreglos consisten en una serie de datos del mismo tipo ubicados en memoria de forma \textbf{contigua}.\\
    \bigskip
    Tienen un tamaño dado por la \textbf{cantidad} de datos del arreglo.\\
    \end{textblock}
\end{frame}

% \begin{frame}[fragile,t]{Ejemplo}
% \begin{frame}[fragile,t]{Ejemplo}
% \small
% \textcolor{verdeuca}{Recorrer un arreglo de numeros calcular el promedio.}
% \begin{verbatim}
% void main() {
%     a[5] = {5,5,3,1,2};
%     
%     
% 
% }
% \end{verbatim}
% \end{frame}

\begin{frame}[fragile,t]{Ejemplo}
\begin{textblock}{87}(110,10)
\footnotesize
\textcolor{verdeuca}{Estructura \texttt{vector}}
\begin{verbatim}
struct vector {
    float x;
    float y;
};
\end{verbatim}
\end{textblock}
\small
\textcolor{verdeuca}{Calcular el módulo de un vector:}
\begin{verbatim}
float modulo2(float x, float y) {
    return sqrt( x * x + y * y);
}
\end{verbatim}
\textcolor{verdeuca}{Calcular el promedio de los módulos de un arreglo de vectores:}
\begin{verbatim}
float promedio(struct vector v[], int size) {
    float acumulador = 0;
    for(int i = 0; i < size; i++) {
        float x = v[i].x;
        float y = v[i].y;
        acumulador = acumulador + modulo2(x,y);
    }
    return acumulador / (float)(size);
}
\end{verbatim}
\end{frame}
% observaciones:
% notacion del vector de estructuras como parametro
% i++
% cast a float

\begin{frame}[fragile,t]{Punteros}
    Los punteros son variables especiales que contienen la dirección de memoria\\ donde se encuentra un dato específico.\\
    \begin{textblock}{70}(10,25)
    \uncover<2->{Se \textbf{declaran} usando un \texttt{*} sobre el nombre de la variable:\\
    \hspace{4cm} \large \texttt{char *p;}\\}
    \bigskip
    \uncover<3->{Se puede \textbf{obtener la dirección de memoria} de un dato usando un \texttt{\&}:\\
    \hspace{4cm} \large \texttt{\&dato}\\}
    \bigskip
    \uncover<5->{Se \textbf{accede} al dato usando un \texttt{*} sobre el nombre de la variable:\\
    \hspace{4cm} \large \texttt{*p} }
    \end{textblock}
    \begin{textblock}{100}(90,40) \only<2->{\includegraphics[scale=0.9]{img/punteroEjemplo-layer1.pdf}} \end{textblock}
    \begin{textblock}{100}(90,40) \only<3->{\includegraphics[scale=0.9]{img/punteroEjemplo-layer2.pdf}} \end{textblock}
    \begin{textblock}{100}(90,40) \only<4->{\includegraphics[scale=0.9]{img/punteroEjemplo-layer3.pdf}} \end{textblock}
    \begin{textblock}{100}(90,40) \only<5->{\includegraphics[scale=0.9]{img/punteroEjemplo-layer4.pdf}} \end{textblock}
\end{frame}

\begin{frame}[fragile,t]{Punteros}
    El tamaño en memoria de un puntero depende del direccionamiento a memoria. \\ En una arquitectura de 64 bits con direcciones de 64 bits, los punteros tienen 8 bytes.
    \begin{textblock}{87}(10,25)
    \textcolor{gray}{Ejemplos:}
    \end{textblock}
    \begin{textblock}{100}(10,35)
    \uncover<2->{\texttt{char a;}\\}
    \uncover<2->{\texttt{char *b;}\\}
    \uncover<6->{\texttt{a = 35;}\\}
    \uncover<7->{\texttt{b = \&a;}\\}
    \uncover<8->{\texttt{*b = 66;}\\}
    \end{textblock}
    \begin{textblock}{100}(35,37) \only<2->{\includegraphics[scale=0.8]{img/punteros-layer1.pdf}} \end{textblock} % 1 byte
    \begin{textblock}{100}(35,37) \only<3-5>{\includegraphics[scale=0.8]{img/punteros-layer2.pdf}} \end{textblock} % dato ?
    \begin{textblock}{100}(35,37) \only<3-6>{\includegraphics[scale=0.8]{img/punteros-layer3.pdf}} \end{textblock} % puntero ?
    \begin{textblock}{100}(35,37) \only<6-7>{\includegraphics[scale=0.8]{img/punteros-layer4.pdf}} \end{textblock} % dato valor
    \begin{textblock}{100}(35,37) \only<7->{\includegraphics[scale=0.8]{img/punteros-layer5.pdf}} \end{textblock} % puntero valor
    \begin{textblock}{100}(35,37) \only<4->{\includegraphics[scale=0.8]{img/punteros-layer6.pdf}} \end{textblock} % dato dir
    \begin{textblock}{100}(35,37) \only<5->{\includegraphics[scale=0.8]{img/punteros-layer7.pdf}} \end{textblock} % puntero dir
    \begin{textblock}{100}(35,37) \only<8->{\includegraphics[scale=0.8]{img/punteros-layer15.pdf}} \end{textblock} % flecha
    \begin{textblock}{100}(35,37) \only<8->{\includegraphics[scale=0.8]{img/punteros-layer16.pdf}} \end{textblock} % nuevo
    \begin{textblock}{100}(10,65)
    \uncover<9->{\texttt{int a;}\\}
    \uncover<9->{\texttt{int *b;}\\}
    \uncover<13->{\texttt{a = 35;}\\}
    \uncover<14->{\texttt{b = \&a;}\\}
    \uncover<15->{\texttt{*b = 66;}\\}
    \end{textblock}
    \begin{textblock}{100}(35,67) \only<9->{\includegraphics[scale=0.8]{img/punteros-layer8.pdf}} \end{textblock} %  4 byte
    \begin{textblock}{100}(35,67) \only<10-12>{\includegraphics[scale=0.8]{img/punteros-layer9.pdf}} \end{textblock} %  dato ?
    \begin{textblock}{100}(35,67) \only<10-13>{\includegraphics[scale=0.8]{img/punteros-layer10.pdf}} \end{textblock} % puntero ?
    \begin{textblock}{100}(35,67) \only<13-14>{\includegraphics[scale=0.8]{img/punteros-layer11.pdf}} \end{textblock} % dato valor
    \begin{textblock}{100}(35,67) \only<14->{\includegraphics[scale=0.8]{img/punteros-layer12.pdf}} \end{textblock} % puntero valor
    \begin{textblock}{100}(35,67) \only<11->{\includegraphics[scale=0.8]{img/punteros-layer13.pdf}} \end{textblock} % dato dir
    \begin{textblock}{100}(35,67) \only<12->{\includegraphics[scale=0.8]{img/punteros-layer14.pdf}} \end{textblock} % puntero dir
    \begin{textblock}{100}(35,67) \only<15->{\includegraphics[scale=0.8]{img/punteros-layer15.pdf}} \end{textblock} % flecha
    \begin{textblock}{100}(35,67) \only<15->{\includegraphics[scale=0.8]{img/punteros-layer17.pdf}} \end{textblock} % nuevo
\end{frame}

\begin{frame}[fragile,t]{Ejemplo}
\small
\textcolor{verdeuca}{Función que suma usando solo punteros:}
\begin{verbatim}
void suma(int* a, int* b, int* c) {
    *c = *a + *b;
}
\end{verbatim}
\textcolor{verdeuca}{Sumar los primeros \texttt{n} números enteros usando la función anterior:}
\begin{verbatim}
int sumaEnteros(int n) {
    int sumatoria = 0;
    for(int i = 0; i < n; i++) {
        suma(&sumatoria, &i, &sumatoria);
    }
    return sumatoria;
}
\end{verbatim}
\end{frame}
% observaciones:
% void
% pasamos el puntero donde escribir el resultado

\begin{frame}[fragile,t]{Ejemplo}
\small
\textcolor{verdeuca}{Calcular el módulo de un vector:}
\begin{verbatim}
float modulo2(struct vector* v) {
    return sqrt( v->x * v->x + v->y * v->y);
}
\end{verbatim}
\textcolor{verdeuca}{Calcular el promedio de los módulos de un arreglo de vectores:}
\begin{verbatim}
float promedio(struct vector* v, int size) {
    float acumulador = 0;
    for(int i = 0; i < size; i++) {
        acumulador = acumulador + modulo2(&v[i]);
    }
    return acumulador / (float)(size);
}
\end{verbatim}
\begin{textblock}{87}(110,10)
\footnotesize
\textcolor{verdeuca}{Estructura \texttt{vector}}
\begin{verbatim}
struct vector {
    float x;
    float y;
};
\end{verbatim}
\end{textblock}
    \begin{textblock}{50}(105,35)
    \begin{block}{\small Equivalencias}
    \small
    \centering
    \textcolor{verdeuca}{Considerar}\\
    \verb|struct vector pepe;|\\
    \vspace{0.1cm}
    \textcolor{verdeuca}{Para escribir \texttt{x} podemos:}\\
    \vspace{0.1cm}
    \verb|(*pepe).x = 45;|\\
    \verb|pepe->x = 45;|\\
    \vspace{0.1cm}
    \textcolor{verdeuca}{Para leer \texttt{x} podemos:}\\
    \vspace{0.1cm}
    \verb|int t = (*pepe).x;|\\
    \verb|int t = pepe->x;|\\
    \end{block}
    \end{textblock}
\end{frame}
% observaciones:
% notacion fechita para punteros

\begin{frame}[fragile]
    \frametitle{Bibliografía}
    \begin{itemize}
     \setlength\itemsep{0.5cm}
      \item[-] \small Brian Kernighan y Dennis Ritchie. “El lenguaje de programación C”, 2da Edición, 1991.\\
      \item[-] \small Mike Banahan, Declan Brady and Mark Doran, “The C Book”, 2da Edición, 1991.\\
      Disponible online: \url{https://publications.gbdirect.co.uk/c_book/}
    \end{itemize}
\end{frame}

\begin{frame}[plain]
    \begin{center}
    \vspace{2cm}
    \huge ¡Gracias!\\
    \vspace{2cm}
%     \normalsize Recuerden leer los comentarios adjuntos\\ en cada clase por aclaraciones.
    \end{center}
\end{frame}

\end{document}
