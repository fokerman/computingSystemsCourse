\documentclass[a4paper,10pt]{article}
\usepackage{geometry}\geometry{top=2cm,bottom=1cm,left=2cm,right=2cm}

\usepackage[T1]{fontenc}
\usepackage[utf8]{inputenc}
\usepackage[spanish,english,es-tabla]{babel}
\selectlanguage{spanish}

\usepackage{amsmath}
\usepackage{amssymb,amsfonts,textcomp}
\usepackage{mdframed}
\usepackage{color}
\usepackage{array}
\usepackage{caption}
\usepackage{supertabular}
\usepackage{hhline}
\usepackage{float}
\usepackage{multicol}
\usepackage{multirow}
% \usepackage[document]{ragged2e}
\usepackage[sfdefault]{AlegreyaSans}
\usepackage[table]{xcolor}
\usepackage[pdftex]{graphicx}
\usepackage[most]{tcolorbox}
\usepackage{hyperref}
\usepackage{tabularx}
\usepackage{multicol}


\usepackage{tikz}
\usetikzlibrary{circuits.logic.US}
\usetikzlibrary{positioning}
% \usepackage{circuitikz}

% % % Apagar numeracion
\pagenumbering{gobble}

% % % PDF warnings
\pdfsuppresswarningpagegroup=1 % para que no te tire el error de que incluis mas de un pdf por hoja
            
% % % TEXTO PDF
\hypersetup{pdftex,
hidelinks=true,
colorlinks=false,
linkcolor=black,
citecolor=black,
filecolor=black,
urlcolor=black, 
pdftitle=,
pdfauthor=,
pdfsubject=,
pdfkeywords=}

% % % TABLAS
\newcolumntype{P}[1]{>{\centering\arraybackslash}p{#1}}
\newcolumntype{C}[1]{>{\centering\arraybackslash}m{#1}}
\newcolumntype{R}[1]{>{\raggedleft\arraybackslash}m{#1}}

% % % ENCABEZADO
\parindent=0pt
\title{\vspace{-13mm}Notación Símbolos y Equivalencias\vspace{-13mm}}
\date{}

\begin{document}
\maketitle

Este documento busca detallar la notación utilizada para representar operaciones lógicas y sus equivalencias.

\section*{Notación de constantes lógicas}

\begin{multicols}{3}
    \begin{center}
    \textbf{Álgebra}\\
    \begin{tabular}{c|p{2.15cm}}
    \small Constantes & \small Descripción\\
    \hline
    \verb|V|  & \small Verdadero o \verb|1|\\
    \verb|F|  & \small Falso o \verb|0|\\
    \end{tabular}
    \end{center}
\columnbreak
    \begin{center}
    \textbf{Lógica digital}\\
    \begin{tabular}{c|p{2.15cm}}
    \small Constantes & \small Descripción\\
    \hline
    \verb|1|              & \small Verdadero o \verb|1|\\
    \verb|0|              & \small Falso o \verb|0|\\
    \end{tabular}
    \end{center}
\columnbreak
    \begin{center}
    \textbf{Código}\\
    \begin{tabular}{c|p{2.15cm}}
    \small Constantes & \small Descripción\\
    \hline
    \verb|True|   & \small Verdadero o \verb|1|\\
    \verb|False|  & \small Falso o \verb|0|\\
    \end{tabular}
    \end{center}
\end{multicols}
    
\section*{Operaciones lógicas}

\begin{multicols}{3}
    \begin{center}
    \textbf{Álgebra}\\
    \begin{tabular}{c|p{3.2cm}}
    \small Operador & \small Descripción\\
    \hline
    $\wedge$           & \small Y lógico {\footnotesize (AND)} \\
    $\vee$             & \small O lógico {\footnotesize (OR)} \\
    $\neg$$p$ o $\sim$$p$  & \small Negado {\footnotesize (NOT de $p$)}\\
    $\bigtriangleup$   & \small Diferencia simétrica {\footnotesize (XOR)}\\
    $\Rightarrow$      & \small Implicación \\
    $\Leftrightarrow$  & \small Si y solo si \\
    \end{tabular}
    \end{center}
\columnbreak
    \begin{center}
    \textbf{Lógica digital}\\
    \begin{tabular}{c|p{2.7cm}}
    \small Operador & \small Descripción\\
    \hline
    \verb|*|    & \small AND \\
    \verb|+|    & \small OR \\
    $\overline{\texttt{x}}$ & \small NOT \verb|x| \\
    $\oplus$    & \small XOR \\
    \end{tabular}
    \end{center}
\columnbreak
    \begin{center}
    \textbf{Código}\\
    \begin{tabular}{c|p{2.7cm}}
    \small Operador & \small Descripción\\
    \hline
    \verb|&&|    & \small AND \\
    \verb.||.    & \small OR \\
    \verb|!x|    & \small NOT \verb|x| \\
    \end{tabular}
    \end{center}
\end{multicols}

\section*{Operaciones en código}
 
\begin{multicols}{3}
    \begin{tabular}{c|p{2.40cm}}
    \small Operador & \small Descripción\\
    \hline
    \verb|+|  & \small Suma           \\
    \verb|-|  & \small Resta          \\
    \verb|*|  & \small Multiplicación \\
    \verb|/|  & \small División entera \\
    \verb|%|  & \small Resto de división\\ 
    \end{tabular}
\columnbreak
    \begin{tabular}{c|p{2.40cm}}
    \small Operador & \small Descripción\\
    \hline
    \verb|==| & \small Igual          \\
    \verb|!=| & \small Distinto       \\
    \verb|>|  & \small Mayor          \\
    \verb|>=| & \small Mayor o igual    \\
    \verb|<|  & \small Menor          \\
    \verb|<=| & \small Menor o igual    \\
    \end{tabular}
\columnbreak
    \begin{tabular}{c|p{2.40cm}}
    \small Operador & \small Descripción\\
    \hline
    \verb|<<| & \small Shift a izquierda \\
    \verb|>>| & \small Shift a derecha   \\
    \hline
    \textasciitilde  & \small NOT bit a bit  \\
    \verb|&|  & \small AND bit a bit  \\
    \verb.|.  & \small OR bit a bit   \\
    \verb|^|  & \small XOR bit a bit  \\
    \end{tabular}
\end{multicols}

Los operandos catalogados como Código pueden variar dependiendo del lenguaje de programación.\\
Sin embargo los mencionados son los más utilizados.

\section*{Simbolos de compuertas lógicas}

\begin{multicols}{3}
    \begin{tabular}{P{4.20cm}}
    \begin{tikzpicture}[circuit logic US, scale=1.4]
    \node [and gate,inputs=nnn] (and1) at (0,0) {};
    \draw (and1.input 1) -- node[at end,left]{A}      ++(left:2mm);
    \draw (and1.input 3) -- node[at end,left]{B}      ++(left:2mm);
    \draw (and1.output)  -- node[at end,right]{A$*$B} ++(right:2mm);
    \end{tikzpicture} \\
    \begin{tabular}{cc|c}
    \texttt{A} & \texttt{B} & \texttt{AND} \\ \hline
    \texttt{0} & \texttt{0} & \texttt{0} \\
    \texttt{0} & \texttt{1} & \texttt{0} \\
    \texttt{1} & \texttt{0} & \texttt{0} \\
    \texttt{1} & \texttt{1} & \texttt{1} \\
    \end{tabular}
    \end{tabular}
\columnbreak
    \begin{tabular}{P{4.20cm}}
    \begin{tikzpicture}[circuit logic US, scale=1.4]
    \node [or gate,inputs=nnn] (or1) at (0,0) {};
    \draw (or1.input 1) -- node[at end,left]{A}      ++(left:2mm);
    \draw (or1.input 3) -- node[at end,left]{B}      ++(left:2mm);
    \draw (or1.output)  -- node[at end,right]{A$+$B} ++(right:2mm);
    \end{tikzpicture} \\
    \begin{tabular}{cc|c}
    \texttt{A} & \texttt{B} & \texttt{OR} \\ \hline
    \texttt{0} & \texttt{0} & \texttt{0} \\
    \texttt{0} & \texttt{1} & \texttt{1} \\
    \texttt{1} & \texttt{0} & \texttt{1} \\
    \texttt{1} & \texttt{1} & \texttt{1} \\
    \end{tabular}
    \end{tabular}
\columnbreak
    \begin{tabular}{P{4.20cm}}
    \begin{tikzpicture}[circuit logic US, scale=1.4]
    \node [not gate,inputs=nnn] (not1) at (0,0) {};
    \draw (not1.input)   -- node[at end,left]{A}                      ++(left:2mm);
    \draw (not1.output)  -- node[at end,right]{$\overline{\text{A}}$} ++(right:2mm);
    \end{tikzpicture} \\
    \begin{tabular}{c|c}
    \texttt{A} & \texttt{NOT} \\ \hline
    \texttt{0} & \texttt{1} \\
    \texttt{1} & \texttt{0} \\
    \end{tabular}
    \end{tabular}
\end{multicols}

\vspace*{0.1cm}

\begin{multicols}{3}
    \begin{tabular}{C{4.20cm}}
    \begin{tikzpicture}[circuit logic US, scale=1.4]
    \node [nand gate,inputs=nnn] (nand1) at (0,0) {};
    \draw (nand1.input 1) -- node[at end,left]{A}                               ++(left:2mm);
    \draw (nand1.input 3) -- node[at end,left]{B}                               ++(left:2mm);
    \draw (nand1.output)  -- node[at end,right]{$\overline{\text{A}*\text{B}}$} ++(right:2mm);
    \end{tikzpicture} \\
    \begin{tabular}{cc|c}
    \texttt{A} & \texttt{B} & \texttt{NAND} \\ \hline
    \texttt{0} & \texttt{0} & \texttt{1} \\
    \texttt{0} & \texttt{1} & \texttt{1} \\
    \texttt{1} & \texttt{0} & \texttt{1} \\
    \texttt{1} & \texttt{1} & \texttt{0} \\
    \end{tabular}
    \end{tabular}
\columnbreak
    \begin{tabular}{C{4.20cm}}
    \begin{tikzpicture}[circuit logic US, scale=1.4]
    \node [nor gate,inputs=nnn] (nor1) at (0,0) {};
    \draw (nor1.input 1) -- node[at end,left]{A}                               ++(left:2mm);
    \draw (nor1.input 3) -- node[at end,left]{B}                               ++(left:2mm);
    \draw (nor1.output)  -- node[at end,right]{$\overline{\text{A}+\text{B}}$} ++(right:2mm);
    \end{tikzpicture} \\
    \begin{tabular}{cc|c}
    \texttt{A} & \texttt{B} & \texttt{NOR} \\ \hline
    \texttt{0} & \texttt{0} & \texttt{1} \\
    \texttt{0} & \texttt{1} & \texttt{0} \\
    \texttt{1} & \texttt{0} & \texttt{0} \\
    \texttt{1} & \texttt{1} & \texttt{0} \\
    \end{tabular}
    \end{tabular}
\columnbreak
    \begin{tabular}{C{4.20cm}}
    \begin{tikzpicture}[circuit logic US, scale=1.4]
    \node [xor gate,inputs=nnnn] (xor1) at (0,0) {};
    \draw (xor1.input 1) -- node[at end,left]{A}           ++(left:2mm);
    \draw (xor1.input 2) -- node[at end,left]{B}           ++(left:2mm);
    \draw (xor1.output)  -- node[at end,right]{A$\oplus$B} ++(right:2mm);
    \end{tikzpicture} \\
    \begin{minipage}{3cm}
    \begin{tabular}{cc|c}
    \texttt{A} & \texttt{B} & \texttt{XOR} \\ \hline
    \texttt{0} & \texttt{0} & \texttt{0} \\
    \texttt{0} & \texttt{1} & \texttt{1} \\
    \texttt{1} & \texttt{0} & \texttt{1} \\
    \texttt{1} & \texttt{1} & \texttt{0} \\
    \end{tabular}
    \end{minipage}
    \end{tabular}
\end{multicols}

\end{document}
